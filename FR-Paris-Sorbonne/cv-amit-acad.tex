%===============================================================================
% Author:   Amit Manohar Manthanwar
% Title:    Research Fellow
% Address:  Aerospace Engineering, Texas A&M University
% Email:    amit@tamu.edu
% Mobile:   +1-979-721-0447
% Phone:    +91-406-639-9631
% WebURL:  https://manthanwar.github.io
%-------------------------------------------------------------------------------
% Author:   Amit Manohar Manthanwar
% aTitle:   Research Associate
% Mailer:   Chemical Engineering, Imperial College London
% E-mail:   amit@imperial.ac.uk
% Mobile:   +1-979-721-0447
% Caller:   +44-796-336-2146
% WebURL:   https://manthanwar.github.io
%-------------------------------------------------------------------------------
%-------------------------------------------------------------------------------
% Author:   Amit Manohar Manthanwar
% Title:    Program Manager
% Address:  Clean Energy Smart Manufacturing Innovation Institute
% Email:    amit@cesmii.org
% Mobile:   +1-979-721-0447
% Phone:    +1-979-458-2552
% WebURL:   https://manthanwar.github.io
%------------------------------------------------------------------------------
% © 2019 Amit Manohar Manthanwar
%-------------------------------------------------------------------------------
% Version Log
%-------------------------------------------------------------------------------
% Date          Author  Version Reason
%-------------------------------------------------------------------------------
% 21-Feb-2007   Amit    1.0     Initial Version
% 12-Apr-2007   Amit    1.1     Imperial Version
% 06-May-2007   Amit    1.2     Imperial Changes
% 27-Aug-2007   Amit    2.0     RasGas Version
% 20-Aug-2008   Amit    200808  Invensys Version
% 10-Mar-2019   Amit    201903  TAMU Aerospace
% 09-Feb-2023   Amit    202302  Imperial
%===============================================================================
\documentclass[10pt]{article}
%\usepackage[showframe]{geometry}
\usepackage{fancyhdr}                           % Convenient URL typesetting
\usepackage{amsmath,amsthm,amssymb,amsfonts}
%\usepackage[usenames]{color}                    % for color text
%\usepackage{colortbl}                           % for colored tabale
\usepackage[dvipsnames]{xcolor}
\usepackage{enumitem}
%\usepackage{calculator}
%\usepackage{epsfig,psfrag}                      % for figure
%\usepackage{epsfig,psfrag}                      % for figure
%\usepackage{graphicx}
%\usepackage{pstricks,pst-all,pst-blur,pst-3dplot}          % for pstricks
%\usepackage{pst-slpe}
%\usepackage{fix-cm}
%\usepackage[round]{natbib}
%\usepackage{eurosym}

\usepackage[breaklinks, colorlinks, citecolor=black, urlcolor=black, allcolors=black,
pdftitle={Amit M. Manthanwar Curriculum Vitae},
pdfauthor={Amit M. Manthanwar},
pdfsubject={amitCV in processed by LaTeX}]{hyperref}

%\usepackage{etaremune}
%====================================================================================
%\paperwidth     = 210mm     %%
%\paperheight    = 297mm     %%
%\hoffset        = -5.4mm   %% offset for printer's left margin
%\voffset        = -10.4mm    %% offset for printer's top margin
%\oddsidemargin  = 0mm       %% Real leftmargin = 1.0 in
%\evensidemargin = 0mm       %% Real leftmargin = 1.0 in
%\textheight     = 260mm     %% 9in
%\textwidth      = 170mm     %% 460 pt = 6 in
%\topmargin      = 0mm       %% Separation between top margin and header
%\headheight     = 0mm       %% Height of the header text
%\headsep        = 0mm       %% Separation between header and main text
%\footskip       = 10mm      %% Separation between footer bottom and main text

% A4 Paper
\iffalse
\paperwidth     = 210mm     %%
\paperheight    = 297mm     %%
\hoffset        = -0.4mm   %% offset for printer's left margin
\voffset        = -5.4mm    %% offset for printer's top margin
\oddsidemargin  = 0mm       %% Real leftmargin = 1.0 in
\evensidemargin = 0mm       %% Real leftmargin = 1.0 in
\textheight     = 250mm     %% 9in
\textwidth      = 160mm     %% 460 pt = 6 in
\topmargin      = -10.4mm       %% Separation between top margin and header
\headheight     = 10mm       %% Height of the header text
\headsep        = 5mm       %% Separation between header and main text
\footskip       = 10mm      %% Separation between footer bottom and main text
\fi
%-------------------------------------------------------------------------------
% Letter Paper
\iftrue
\paperwidth     =  8.5in     %%
\paperheight    = 11.0in     %%
\hoffset        = -0.0in   %% offset for printer's left margin
\voffset        = -0.5in    %% offset for printer's top margin
\oddsidemargin  =  0.0in       %% Real leftmargin = 1.0 in
\evensidemargin =  0.0in       %% Real leftmargin = 1.0 in
\textheight     =  9.0in     %% 9in
\textwidth      =  6.5in     %% 460 pt = 6 in
\topmargin      =  0.0in       %% Separation between top margin and header
\headheight     =  0.2in       %% Height of the header text
\headsep        =  0.3in       %% Separation between header and main text
\footskip       =  0.5in      %% Separation between footer bottom and main text
\fi
%-------------------------------------------------------------------------------
%-------------------------------------------------------------------------------
\setlength{\parindent}{0pt}
\setlength{\parskip}{5pt}
%\setlength{\parsep}{20pt}

%\setlength{\fboxsep}{10pt}
\newcommand{\parSpacing}{5pt}

\DeclareUnicodeCharacter{20AC}{\EUR{}}

\newenvironment{itemizeZero}{\begin{itemize}
%\setlength{\parskip}{0pt}%
\setlength{\itemsep}{0pt}
%\setlength{\leftmarginii}{20mm} % default 2.2em
}
{\end{itemize}}

\newenvironment{enumerateZero}{\begin{enumerate}
%\setlength{\parskip}{0pt}%
\setlength{\itemsep}{0pt}
}
{\end{enumerate}}

\newenvironment{enumerateZeroR}{\begin{etaremune}
%\setlength{\parskip}{0pt}%
\setlength{\itemsep}{0pt}
}
{\end{etaremune}}



%-------------------------------------------------------------------------------
%% 72pt = 1in = 25.4mm
% A4 Paper      = 210.0 x 297.0 mm  = 8.27 x 11.69 in   = (1in+6.27in+1in x
% Letter Paper  = 215.9 x 279.4 mm  = 7.50 x 11.00 in
%====================================================================================
%\renewcommand{\familydefault}{\sfdefault}
%-------------------------------------------------------------------------------
\fancyhf{} % Clear all fields
\renewcommand{\headwidth}{\textwidth}
\renewcommand{\headrulewidth}{0.0pt}
\renewcommand{\footrulewidth}{0.0pt}
\fancyfoot[L]{}
\fancyfoot[R]{}
\fancyfoot[C]{\footnotesize{\thepage\ of \pageref{LastPage}}}



%-------------------------------------------------------------------------------
\newcommand{\etal}{et al.}
%====================================================================================
\title{\vspace{-15mm}\huge{\textbf{Amit Manohar Manthanwar\\[-2mm]}}}
\author{
%\small{Research Fellow, Aerospace Engineering, Texas A\&M University}\\[1mm]
%\small{1617 Research Parkway, College Station, Texas 77843 }\\[1mm]
\small{601 Luther Street West, Unit 1021-D, College Station, Texas 77840 }\\[0mm]
\small{+1.979.721.0447 \;\;$\bullet$\;\; amit@tamu.edu \;\;$\bullet$\;\; \href{https://manthanwar.github.io}{https://manthanwar.github.io}}}
%\date{\small{Updated: March 21, 2019}}
\date{}%\small{Updated: \today}}
%\date{}
%====================================================================================
\begin{document}
%====================================================================================
%====================================================================================
%\maketitle
\pagestyle{fancy}
%\thispagestyle{firstpage}
%\thispagestyle{empty}% \fancyfoot[R]{}
%\setlength{\baselineskip}{12pt}

%====================================================================================
%\vspace{-15mm}
\begin{center}
\huge{\textbf{Amit Manohar Manthanwar\\[2mm]}}

%\small{Research Fellow, Aerospace Engineering, Texas A\&M University}\\[1mm]
%\small{1617 Research Parkway, College Station, Texas 77843 }\\[1mm]
\small{3 Sterling Apartment, Ram Nagar Road, Warje Malwadi, Pune, Maharashtra 411058, India}\\[2mm]
\small{+91.853.081.3398 \;\;$\bullet$\;\; manthanwar@hotmail.com \;\;$\bullet$\;\; \href{https://manthanwar.github.io}{https://manthanwar.github.io}}

\end{center}
%====================================================================================
%\rule[10pt]{\textwidth}{0.4pt}
%\vspace{-20pt}
%\pcline[linewidth=0.4pt](0,10pt)(\textwidth,10pt)
%\pcline[linewidth=0.4pt](0,8.5pt)(\textwidth,8.5pt)
%\vspace{-4pt}
%\vspace{\parSpacing}
%\renewcommand{\abstractname}{}
%\begin{abstract}
\bigskip

%\section*{Objective}
The objective is to pursue research, development and deployment agenda in the area of systems engineering with specific intentions to contribute to the development of (a) renewable energy systems; (b) optimization and control theory; (c) smart manufacturing and Industry 4.0 digital platform ecosystem and underlined operations/information technologies; (d) engineered systems that can adapt/recover from expected/unexpected impact of uncertainty; (e) innovative research programs, knowledge and technology transfer activities; and (f) strategic partnerships with government, industry, academia and national labs to revolutionize energy and manufacturing sectors.

%of systematic activities of
%industrial automation
%in the area of environmentally benign, cost-effective industrial strategies
%====================================================================================


\section{Experience}
\begin{itemize}[itemsep=-1mm]
\item 2019---{\color{white}{0000}} \;\; Research Analyst, Freelance Technology Consultancy
\item 2019---2019 \;\; Research Fellow, Aerospace Engineering, Texas A\&M University
\item 2017---2019 \;\; Program Manager, US DOE Smart Manufacturing Insitute, Texas A\&M University
%Southern Regional Manufacturing Center, CESMII% Clean Energy Smart Manufacturing Innovation Institute
%\item 2017---2019 Program Manager, Texas A\&M Energy Institute, Texas A\&M University
\item 2016---2017 \;\; Research Associate, Texas A\&M Energy Institute, Texas A\&M University
\item 2012---2016 \;\; Research Associate, Centre for Process Systems Engineering, Imperial College London
\item 2009---2012 \;\; Assistant Professor, Instrumentation and Control, College of Engineering Pune
%\item 2009---2010 \;\; Lecturer, Instrumentation and Control, College of Engineering
\item 2007---2009 \;\; Advanced Process Control Software Developer, Invensys (now Schneider Electric)
\item 2007---2007 \;\; Optimization and Advanced Process Control Consultant, RasGas Company Limited
\item 2005---2006 \;\; Web Technology Developer, Chemical Engineering, Imperial College London
\item 2003---2006 \;\; Research Assistant, Centre for Process Systems Engineering, Imperial College London
\item 2001---2003 \;\; Research and Teaching Assistant, Chemical Engineering, Illinois Institute of Technology
%\item 2001---2003 Research Assistant, Chemical Engineering, Illinois Institute of Technology, Chicago
\end{itemize}


\section{Honors}
\begin{itemize}[itemsep=-1mm]
\item 2018 \;\; Smart Manufacturing of Chemical Processing, Department of Energy Grant
\item 2013 \;\; Best Energy Exhibit Award, UK Energy Research Centre
\item 2011 \;\; Institution of Chemical Engineers (IChemE) Journal's Top Reviewers Award
\item 2008 \;\; Invensys Award for Connoisseur{\texttrademark} Software Development (now Schneider Electric) % Version  14.3)
\item 2002 \;\; Excellence in Teaching Award, Illinois Institute of Technology
\end{itemize}
%====================================================================================

\section{Expertise}
\begin{itemize}
\item Research, Development and Deployment: Contributed to the design and operation of process, energy, and structural systems. Theoretical and algorithmic contributions made in the area of model predictive control of linear and classes of nonlinear systems using convex optimization and multi-parametric programming. Pioneering contributions made to the inverse optimal control and the economic value of control strategies. Experimental contributions made in the area of fuel cell systems. Contributed to the development of smart manufacturing of chemical processing and hybrid additive manufacturing.
\item Technology Transfer: Developed commercial industrial automation and advanced process control algorithms, tools, software products and solutions suits used by some of the leading industries.
\item Knowledge Transfer: Taught undergraduate and graduate level process systems engineering courses. Delivered workshops on advanced industrial automation and control technology.%Supervised student projects.
\end{itemize}
%professional continual education


\section{Education}
\begin{itemize}
\item Ph.D., Chemical Engineering, Imperial College London (thesis pending)
\begin{itemize}[nosep]
\item Thesis: Multiscale design, robust optimisation and control of fuel cell energy systems.
\item Committee: Efstratios Pistikopoulos, Nilay Shah, Anthony Kucernak, Constantinos Pantelides%, Morten Hovd
\item This work developed a novel design, system integration and control automation framework using multi-parametric programming and robust optimisation techniques, two prominent approaches to decision making under uncertainty.  Research carried out contributed to the experimental understanding of fuel cell phenomena and theoretical advancement of multi-parametric model predictive control. From the practical standpoint, the focus was to develop efficient fuel cell controllers. Towards that goal this work developed two novel experimental test facilities. The first facility helps understand inner workings of an operating polymer electrolyte membrane fuel cell by developing in situ experimental maps of spatiotemporally resolved temperature distributions using a combined array of 140 temperature sensors placed on both anode and cathode surfaces where the reaction occurs. These results were combined with the pressure profiles using reactant imaging data and water distributions using neutron tomography. This work enables to validate the high fidelity mathematical models of the fuel cell system and underlined subsystem. The second facility is a state-of-the-art fully automated, fully integrated fuel cell power plant built to develop, deploy and demonstrate advanced control automation technologies and the proposed system integration framework. This pilot plant was as one of the smart manufacturing testbeds.
\end{itemize}
\item M.S., Chemical Engineering, Illinois Institute of Technology
\begin{itemize}[nosep]
  \item Thesis: Tuning of model predictive controllers
  \item Committee: Donald Chmielewski, Geoffrey Williamson, Satish Parulekar, Fouad Teymour
  \item This work presents a systematic approach for the tuning of linear model predictive controllers based on an economic interpretation of the computationally attractive covariance constrained control problem. Work addressed Kalman's long standing question of inverse optimal control. Additionally, work contributed to the sensor/actuator hardware selection and placement problem. This work has been incorporated in the book titled `Smart process plants: software and hardware solutions for accurate data and profitable operations'.
\end{itemize}
\item  B.E., Petrochemical Engineering, Maharashtra Institute of Technology
\begin{itemize}[nosep]
\item Thesis: Plant-wide control
  \item Committee: Prashant Shevgaonkar, Datta Dandge, Sanjay Deshmukh %Vishwakarma Institute of Technology
  \item This work investigated plant-wide control as an innovative approach to design control structure, which effectively controls the important variables of a multiprocess plant and coordinates all inter connected unit operations.
\end{itemize}
\end{itemize}



\section{Research Interests}
\begin{itemize}[nosep]
\item Systems Engineering: mathematical optimisation; covariance bounded control; model predictive control

\item Synthesis and Design: experimental investigation of energy systems; environmentally conscious process design; theoretical study of traditional and renewable energy systems; static and dynamic structures

\item Smart Manufacturing: cloud platform reference architecture; embedded systems; digital twinning and automation; contextualization and workflow orchestration; networking, communication and security
\end{itemize}


\section{Teaching Interests}

%\subsection{Courses and Curricula Development}
\begin{itemize}[nosep]
\item Courses taught: numerical techniques; process design; advanced process optimization; advanced process control; energy systems engineering
\item Curricula under preparation: fundamentals of smart manufacturing
\end{itemize}


\subsection{Teaching Philosophy}
\begin{itemize}
\item Objective:
\begin{itemize}[nosep]
\item Develop certificates and degree programs to ensure relevance, quality, and compliance
\item Develop competencies and achieve individual-centric educational goals
\item Encourage curiosity, creativity and active learning interactions/collaboration
\item Communicate high expectations of deliverables and time management
\item Target successful completion and expand career opportunities
\item Create/improve effective world-class learning environment of pedagogical excellence
\item Advance \emph{guru–shishya} (teacher-disciple) tradition and build lasting relationships %based on mutual respect and trust
\end{itemize}

\item Pedagogy:
\begin{itemize}[nosep]
\item Knowledge transfer using application driven hands-on training approaches
\item Inculcate habits of clear critical thinking, reading and problem solving
\item One-to-one, one-to-many and many-to-many student mentoring, advising support network
\item Adapt traditional and contemporary learning and teaching aids
\item Bridge knowledge gaps by introducing multidisciplinary topics
\end{itemize}

\item Assessment:
\begin{itemize}[nosep]
\item Assess knowledge imparted and skills developed thorough assignments, exams and  projects
\item Provide feedback through group/self-study activities, case studies
\item Receive peer review feedback through scholarly publications and competitions
\item Assess teaching strategies, alternative pedagogies, program offerings, and innovations
\end{itemize}
\end{itemize}



\section{Publications}
\subsection{Journal Papers}
\begin{enumerate}[nosep]
\item T. Lopes,  A.M. Manthanwar, A.R. Kucernak, \etal, ``Spatially resolved oxygen reaction, water, and temperature distribution: experimental results as a function of flow field and implications for polymer electrolyte fuel cell operation'', Applied Energy, vol 252, art 113421, 2019

%\item E.N. Pistikopoulos, N.A. Diangelakis, A.M. Manthanwar, ``Towards the integration of process design, control and scheduling: are we getting closer?'', Computer Aided Chemical Engineering, vol 37, pp 41-48, 2015

\item N.A. Diangelakis, A.M. Manthanwar, E.N. Pistikopoulos, ``A framework for design and control optimisation: application on a CHP system'', Computer Aided Chemical Engineering, vol 34, pp 765-770, 2014

\item A.M. Manthanwar, V. Sakizlis, V. Dua, E.N. Pistikopoulos, ``Robust model-based predictive controller for hybrid system via parametric programming'', Computer Aided Chemical Engineering, vol 20, pp 1249-1254, 2005

\item D.J. Chmielewski and A.M. Manthanwar, ``On the tuning of predictive controllers: inverse optimality and the minimum variance covariance constrained control problem'', Industrial Engineering and Chemistry Research, vol 43 (24), pp 7807--7814, 2004

\item Jui-Kun Peng, A.M. Manthanwar, D.J. Chmielewski, ``On the tuning of predictive controllers: the minimum back-off operating point selection problem'', Industrial Engineering and Chemistry Research, vol 44 (20), pp 7814--7822, 2004

\end{enumerate}


\subsection{Conference Proceedings}
\begin{enumerate}[nosep]
\item B. Botcha, Z. Wang, S. Rajan, N. Gautam, S.T.S. Bukkapatnam, A.M. Manthanwar, S. Miller, D. Schneider and P. Korambath, ``Implementing the transformation of discrete part manufacturing systems into smart manufacturing platforms'', ASME Proceedings Manufacturing Equipment and Systems, 2018

\item N.A. Diangelakis, A.M. Manthanwar, E.N. Pistikopoulos, ``Towards the integration of process design, control and scheduling: Are we getting closer?'', Proceedings of the European Symposium on Computer Aided Process Engineering, 2015

\item N.A. Diangelakis, A.M. Manthanwar, E.N. Pistikopoulos, ``A Framework for Design and Control Optimisation: Application on a CHP System'', Proceedings of the European Symposium on Computer Aided Process Engineering, 2014

\item N.A. Diangelakis, A.M. Manthanwar, E.N. Pistikopoulos, ``A Framework for Design and Control Optimisation: Application on a CHP System'', Proceedings of the 8th International Conference on Foundations of Computer-Aided Process Design - FOCAPD, July 13-17, Cle Elum, Washington, 2014

\item A.M. Manthanwar, V. Sakizlis, E.N. Pistikopoulos, ``Robust Parametric Predictive Control Design for Polytopically Uncertain Systems'', IEEE Proceedings of the American Control Conference, Portland, Oregon, USA, pp.3994–3999, 2005.

\item A.M. Manthanwar, V. Sakizlis, E.N. Pistikopoulos, ``Design of Robust Parametric MPC for Hybrid Systems'', Proceedings of the IFAC World Congress, Prague, Czech Republic, 2005

\item A.M. Manthanwar, V. Sakizlis, V. Dua, E.N. Pistikopoulos, ``Robust model-based predictive controller for hybrid system via parametric programming'', Proceedings of the ESCAPE, Portugal, 2005

\item A.M. Manthanwar, V. Sakizlis, E.N. Pistikopoulos, ``Explicit MPC for Hybrid Systems'', Proceedings of the National Conference on Control and Dynamical Systems, Indian Institute of Technology, Mumbai, 2005

\item J.K. Peng, A.M. Manthanwar, D.K. Chmielewski, ``Convex Methods in Actuator Placement'', IEEE Proceedings of the American Control Conference, Fairbanks, Alaska, USA, vol. 6, pp. 4309–4314, 2002
\end{enumerate}

\subsection{Book Chapters}
\begin{enumerate}[nosep]
\item Miguel J. Bagajewicz and Donald J. Chmielewski, ``Value of Control Strategies'', In book -- Smart Process Plants: Software and Hardware Solutions for Accurate Data and Profitable Operations -- Data Reconciliation, Gross Error Detection, and Instrumentation Upgrade, McGraw-Hill Education, First edition, 2009.
\end{enumerate}

\section{Other Contributions}
\subsection{Conference Presentations}
\begin{enumerate}[nosep]
\item A.M. Manthanwar and  E.N. Pistikopoulos, Smart Manufacturing Framework for the Production of Hydrogen Energy, AIChE Spring Meeting, 2016 (Panel discussion on Smart Manufacturing)
%(Member on the panel discussion on Smart Manufacturing)

\item A.M. Manthanwar and  E.N. Pistikopoulos, Deployment of a Smart Manufacturing Framework for the Production of Hydrogen Energy, Texas A\&M Conference on Energy, 2016

\item A.M. Manthanwar and E.N. Pistikopoulos, ``Deployment of explicit model predictive controller onboard a mini fuel cell vehicle'', AIChE Annual Meeting, 2015

\item A.M. Manthanwar and E.N. Pistikopoulos, ``A generic framework for the robust control of fuel cell energy system'', AIChE Annual Meeting, 2015

\item A.M. Manthanwar, T. Lopes, S.C. Atkins, A.R. Kucernak and E.N. Pistikopoulos, ``Novel in situ experimental technique to understand inner workings of a polymer electrolyte membrane fuel cell'', AIChE Annual Meeting, 2015

\item A.M. Manthanwar, E.N. Pistikopoulos and A.R. Kucernak, ``Multi-Scale Experimental Analysis, Robust Optimisation and Explicit Model Predictive Control of Fuel Cell Energy Systems'', The Hydrogen and Fuel Cell Researcher Conference, Birmingham, UK, 15-17 December, 2014
\end{enumerate}

\subsection{Conference Posters}
\begin{enumerate}[nosep]
\item Gerald Ogumerem, Amit M. Manthanwar, Babak Rahrov, Efstratios Pistikopoulos, Fuel Cell Hybrid Vehicle Multi-Parametric Controller, Texas A\&M Conference on Energy, 2016

\item Amit M. Manthanwar, T. Lopes, Kucernak A.R., and E.N. Pistikopoulos, ``Investigation of Hotspots and Uncertainty in Operating Fuel Cell'', Gordon Research Conference, 2016

\item Amit M. Manthanwar and E.N. Pistikopoulos, ``Automation of Fuel Cell Energy Systems'', Texas A\&M Energy Institute, Research Symposium, College Station, Texas, 2015

\item Amit M. Manthanwar, ``The High Level Policy Conference On Sustainable Energy'', The European Commission, Brussels, Brussels, Belgium, 24–26 June, 2014

\item C.V. Kulkarni, Amit M. Manthanwar and B.K. Kakati, ``Novel  Computational and Experimental Approaches for the Optimal Design and Synthesis of Self-Assembling Molecules'', Proceeding of Professor CNR Rao 80th Birthday Symposium, The Royal Society of Chemistry, Burlington House, London, UK, 23–24 June, 2014

\item Amit M. Manthanwar, T. Lopes, S.C. Atkins, A.R. Kucernak and E.N. Pistikopoulos, ``Advances in Fuel Cell Automation and Control'', The Hydrogen and Fuel Cell Researcher Conference, Birmingham, UK, 16-18 December, 2013

\item Amit M. Manthanwar and E.N. Pistikopoulos, "Robust Model Predictive Controller For Fuel Cell System", International School on Energy Systems, Organized by Jülich Research Centre at Kloster Seeon, Germany, 1-5 October, 2012

\item Amit M. Manthanwar, ``Robust MPC for Hybrid System via Parametric Programming'', IChemE CAPE PhD Research Symposium, UCL, London, 2006
\end{enumerate}

\subsection{Invited Seminars}
\begin{enumerate}[nosep]
\item Dean Schneider, Amit M. Manthanwar, Scott Miller, Mahendra Sunkara and Mark McGinley, ``Regional Seminar on Smart Manufacturing'', University of Louisville, 2210 South Brook St., Shumaker Research Bldg. Room 139, Belknap Campus, Louisville, KY, 40208, November 29, 2018

\item Dean Schneider, Amit M. Manthanwar, Scott Miller, DeDe Griffith, ``Regional Seminar on Smart Manufacturing'', Northwest Louisiana Technical College
9500 Industrial Drive, Minden, LA 71055, July 24, 2018

\item Amit M. Manthanwar, Dean Schneider, Scott Miller, Srinivas Palanki, ``Regional Seminar on Smart Manufacturing'', Lamar University - Center for Innovation, Commercialization and Entrepreneurship (CICE) Building, 1003 E Florida Ave., Beaumont, TX 77705, May 2, 2018

\item Amit M. Manthanwar, Dean Schneider, Scott Miller, Gus Kousoulas, ``Regional Seminar on Smart Manufacturing'', Louisiana State University, Digital Media Center, 340 East Parker Boulevard, Baton Rouge, LA 70803, April 17, 2018

\item Amit M. Manthanwar, ``Smart Manufacturing in Process Technology: Advancing Operator Impact'', NAPTA Winter Meeting, Isle of Capri Conference Center - Lake Charles, LA, February 23, 2018

\item Amit M. Manthanwar, Dean Schneider, Scott Miller, Ramanan Krishnamoorti, ``Regional Seminar on Smart Manufacturing'', University of Houston - Center for Innovation and Partnership at Energy Research Park, 5000 Gulf Freeway, Building 4, Houston, TX 77023, December 11, 2017

\item Amit M. Manthanwar, Dean Schneider, Scott Miller, Regional Seminar on Smart Manufacturing, Texas A\&M Engineering Experimentation Station - Houston Office
15835 Park Ten Place, Suite 160, Houston, TX 77084, December 11, 2017

\item Amit M. Manthanwar, Dean Schneider, Scott Miller, Eileen Clements, ``Regional Seminar on Smart Manufacturing'',  University of Texas Arlington Research Institute (UTARI), 7300 Jack Newell Boulevard South, Fort Worth, TX 76118, December 7, 2017

\item Amit M. Manthanwar, Dean Schneider, Scott Miller, ``Regional Seminar on Smart Manufacturing'', Texas A\&M University, Suite 404 Rudder Tower, College Station, TX 77843-1232, November 29, 2017

\item Amit M. Manthanwar, ``Envisioning Hydrogen Economy: Vision and Goals of Multi-Scale Fuel Cell Energy Systems Engineering Laboratory'', Texas A\&M Energy Institute, Texas A\&M University, April 15, 2016

\item Amit M. Manthanwar, ``On the Industrial Process Automation: Control Theory, Applications and Beyond'', Department of Computing, Imperial College London, UK, February 20, 2013

\item Amit M. Manthanwar, ``On the Trends in Process Automation'', The Institution of Engineers, Pune Centre, Maharashtra India, October 15, 2010

\item Amit M. Manthanwar, ``Industrial Process Control: Research Vision and Goals'', RasGas Company Limited, Texas A\&M University at Qatar, April 19, 2007
\end{enumerate}

\subsection{Technical Reports}
\begin{enumerate}[nosep]
\item Amit M. Manthanwar, ``On the Industrial Process Identification'', Technology Report, Invensys Development Centre, Invensys Process Automation, 2008

\item Amit M. Manthanwar, ``Towards Optimal Design and Optimal Operation of Cement Kilns'',  Technology Report, Invensys Development Centre, Invensys Process Automation, 2008

\item Amit M. Manthanwar, Michael Lewis, James HyoungKi Lee and Soroya Puranitee, Global Energy Policy: Social, Political and Technological Impact on Environment and Economics, Illinois Institute of Technology, Chicago, 2003
\end{enumerate}

\subsection{Workshops}
\begin{enumerate}[nosep]
\item Workforce Development and Education Workshop, DOE Clean Energy Smart Manufacturing Innovation Institute, April 18-19, 2018

\item American Institute of Chemial Engineers (AIChE), Center for Chemical Process Safety (CCPS) and Safety and Chemical Engineering Education (SAChE) Faculty Workshop at Dow Chemical Company in Freeport, TX, 21-23 June, 2016

\item ARM mbed Educator’s Workshop, ARM Incorporation, The ARM University Program, Cambridge, 17-18 June, 2013

\item The Fundamentals of PEM Fuel Cells, Enlargement and Integration Action Workshop, Institute for Energy and Transport, July 2012

\item International School on Energy Systems, Kloster Seeon, Germany, October 1-5, 2012
\end{enumerate}

\section{Grants and Funding}
%\subsection{Funding Proposals}
\begin{enumerate}[nosep]
\item Amit M. Manthanwar (PI), Materich Germany, Gmina Osiek Poland, Platform INS Netherlands, ``Art-driven Computer Training Now (ACTNOW): a cultural and creative approach for gender-responsive STEAM education'', EU HORIZON-CL2-2023-HERITAGE-01-08 (Under Review EUR 605750)
\item Amit M. Manthanwar (PI), Materich Germany, Gmina Osiek Poland, Platform INS Netherlands, ``Societal Transformations Through Art (START): Promoting cultural literacy through arts education to foster social inclusion'', EU HORIZON-CL2-2023-HERITAGE-01-07 (Under Review EUR 361250)
\item Amit M. Manthanwar (PI), Materich Germany, Water Insight Netherlands, Gdansk University of Technology, Institute of Oceanology Polish Academy of Sciences, Bayer AG, STAR - WiK Spółka, ``Neutralisation of Pathogens in Baltic Polish Water Ecosystem'', EU HORIZON-CL6-2022-ZEROPOLLUTION-01-04 (Rejected EUR 2.8M)
\item Amit M. Manthanwar (PI), Stratos Pistikopoulos (TAMU PI), Larry Megan (Praxair PI), Wayne Bequette (RPI PI), Ashok Rao (AspenTech PI), Jim O’Rourke (OSIsoft PI), Michael Baldea (UT PI), Dimitrios Georgis (PSE PI), Pete Sharp (Emerson PI), et al., ``Smart manufacturing for chemical processing: energy efficient operation of air separation unit'', Department of Energy, The Clean Energy and Smart Manufacturing Innovation Institute, 2018    (Awarded USD 2.7M)

\item Amit M. Manthanwar (PI), Daniel Chen (Lamar PI), Keith Biggers (TAMU PI), Wayne Bequette (RPI PI), Jim O’Rourke (OSIsoft PI), ``Development of Decision Support System Powered by the Smart Manufacturing Platform Ecosystem'', Department of Energy, The Clean Energy and Smart Manufacturing Innovation Institute, 2018 (Rejected USD 1.3M)

\item Amit M. Manthanwar (Research Investigator), James Davis (UCLA PI), Thomas Edgar (UT PI) Christodoulos A. Floudas (TAMU PI), Efstratios N. Pistikopoulos (TAMU CI), ``Testbed for the Smart Automation of Sustainable Fuel Cell Energy Generation'', Department of Energy, The Clean Energy and Smart Manufacturing Innovation Institute, 2016 (Awarded USD 140M to UCLA that created Regional Manufacturing Center at TAMU)
    %

\item Amit M. Manthanwar (Research Investigator) and Efstratios N. Pistikopoulos (PI), ``A Novel Multi-Parametric Optimization and Control Framework for the Efficient Operation of Multi-Scale Hydrogen Energy Networks'', Shell, 2015 (Awarded USD 485K)

\item Amit M. Manthanwar (PI), Thiago Lopes (CI), Anthony R.J. Kucernak (CI), and Efstratios N. Pistikopoulos (CI), ``In Situ Measurement and Optimal Performance Monitoring System for Polymer Electrolyte Fuel Cell'', EPSRC Internal Funds – Manufacturing Futures Kick-start, October 2012 (Awarded GBP 15K)

\item Amit M. Manthanwar (Research Investigator), Thiago Lopes (Research Investigator), Anthony R.J. Kucernak (CI), and Efstratios N. Pistikopoulos (PI), ``A Generic Framework for the Development of Efficient Fuel Cell Controllers, EPSRC, The SUPERGEN Hydrogen and Fuel Cells Challenge'', August 16. 2012 (Rejected GBP 890K)

\item Amit M. Manthanwar (CI), S.D. Agashe (PI), et al., ``Development of
Process Automation Laboratory'', Department of Science \& Technology, 2010 (Awarded INR 300M)

\item Amit M. Manthanwar (PI), ``A Proposal for the Advanced Process Control Remote Laboratory'', Ministry of Human Resource Development, 2011 (Awarded INR 780M)

\end{enumerate}


\subsection{Competitions}
\begin{enumerate}[nosep]
\item Amit M. Manthanwar, ``Robust Optimisation and Control Automation of Fuel Cells: Advances in Clean Energy Systems'', Research Competition at Westminster, UK Parliament, March 9, 2015

\item Amit M. Manthanwar, B. Patel, C. Sheridan and C. Stavrou, ``Innovation through Catalysis'', UK Energy Young Entrepreneurs Scheme (EnergyYES) Competition, Won Second Place, Alstom Energy,  Rugby, UK, May 22–23, 2014

\item Amit M Manthanwar, A. Morrison, B. Patel, D. Barker and C. Bannar-Martin, ``Biomass Powered Combined Heat and Electricity Source for Use in Rural Communities'', Won Second Prize, Alliance of Global Technological Universities, GlobalTech Energy Challenge, 2013

\item Amit M. Manthanwar, ``Demonstration of Integration of Renewable Energy Sources'', Received Best Energy Exhibit Award, Annual Energy CDT Conference, University of Manchester, 28-30 Nov 2012
\end{enumerate}


\subsection{Software Developed}
\begin{enumerate}[nosep]
\item \TeX\ Postscript package \textsf{pst-flags} for graphic design, 2022
\item Online Real-time GIS based disease tracking and resource management system, 2021
\item Online digital platform with bigdata visualization and management system. 2020
\item Model Predictive Control \textsf{Connoisseur}\texttrademark\ , the advanced process control performance suite, Invensys (now Schneider Electric), version 15.3, 2008
%Process Operations Management
\item Integration bridge for Foxboro Intelligent Automation (I/A) Distributed Control System (DCS), Invensys Operations Management, (now Schneider Electric), version 1.4, 2009
\item Enterprise information system for admission application processing and management, Pune University PhD admissions, 2009
\end{enumerate}

\subsection{Hardware Developed}
\begin{enumerate}[nosep]
\item ARM microcontroller based hardware prototype for testing system-on-chip optimization and control strategies, Department of Chemical Engineering, Imperial College London, 2015

\item Highly instrumented and fully automated fuel cell power plant test facility, Department of Chemical Engineering, Imperial College London, 2014

\item In situ experimental test facility for characterizing thermal behaviour of fuel cells, Department of Chemistry, Imperial College London, 2013
\end{enumerate}
%====================================================================================

\section{Professional Activities}

%\subsection{Editorial Positions}
%\subsection{Service}
%\subsection{Consulting}

\subsection{Reviewer}
\begin{itemize}[nosep]
\item AIChE Journal
\item Chemical Engineering Science
\item Springer Energy Systems
\item Computers and Chemical Engineering
\item European Journal of Control
\item IEEE Transaction on Control System Technology
%\item European Control Conference
\item Education for Chemical Engineers
\end{itemize}

%\subsection{Outreach Activities}

\subsection{Professional Societies}
\begin{itemize}[nosep]
\item American Institute of Chemical Engineers (AIChE)
\item Institute of Electrical and Electronics Engineers (IEEE)
\end{itemize}


\subsection{References}
\begin{itemize}
\item Professor Nilay Shah
\begin{itemize}[nosep]
\item Department of Chemical Engineering, Imperial College London
\item South Kensington Campus, London SW7 2AZ, United Kingdom
\item n.shah@imperial.ac.uk \;$\bullet$\; +44.207.594.6621
\end{itemize}

\item Professor Anthony Kucernak
\begin{itemize}[nosep]
\item Department of Chemistry, Imperial College London
\item South Kensington Campus, London SW7 2AZ, United Kingdom
\item anthony@imperial.ac.uk \;$\bullet$\; +44.207.594.6621
\end{itemize}

\item Professor James Davis
\begin{itemize}[nosep]
\item Department of Chemical Engineering, University of California Los Angeles
\item 2333 Murphy Hall, 7400 Boelter Hall, Los Angeles, CA 90095, USA
\item jdavis@oit.ucla.edu \;$\bullet$\; +1.310.206.0011
\end{itemize}

\item Professor Efstratios Pistikopoulos
\begin{itemize}[nosep]
\item  Director of Energy Institute, Texas A\&M University
\item 1617 Research Parkway, College Station, TX 77843, United States
\item stratos@tamu.edu \;$\bullet$\; +1.979.458.0259
\end{itemize}

\item Professor Donald J. Chmielewski
\begin{itemize}[nosep]
\item Department of Chemical Engineering, Illinois Institute of Technology
\item 10 West 33rd Street, Perlstein Hall, Suite 224, Chicago, IL 60616
\item chmielewski@iit.edu \;$\bullet$\;  +1.312.567.3537
\end{itemize}

\item Professor Robert E. Skelton
\begin{itemize}[nosep]
\item Department of Aerospace Engineering, Texas A\&M University
\item 701 H.R. Bright Building 3141, College Station, TX 77843
\item bobskelton@tamu.edu \;$\bullet$\;  +1.979.845.3947
\end{itemize}

\item Dr Larry Megan
\begin{itemize}[nosep]
\item Director, Praxair Digital (now Linde Digital)
\item 175 East Park Dr Tonawanda NY 14150
\item larry\_megan@praxair.com \;$\bullet$\;  +1.716.879.2061
\end{itemize}

\item Dr Haresh Malkani
\begin{itemize}[nosep]
\item Chief Technology Officer, Clean Energy Smart Manufacturing Innovation Institute
\item 175 East Park Dr Tonawanda NY 14150
\item larry\_megan@praxair.com \;$\bullet$\;  +1.716.879.2061
\end{itemize}

%\item Mr Ashok Rao
%\begin{itemizeZero}
%\item Senior Director, Product Management at Aspen Technology
%\item 2500 Citywest Blvd., Ste 1600, Houston, TX 77042
%\item ashok.rao@aspentech.com \;$\bullet$\;  +1.281.504.3150
%\end{itemizeZero}


%\item Professor James Davis
%\begin{itemizeZero}
%\item Vice Provost, Information Technology \& Chief Academic Technology Officer, UCLA
%\item 5308 Math Sciences, Los Angeles, CA 90095-1557
%\item jdavis@oit.ucla.edu \;$\bullet$\;  +1.310.206.0011
%\end{itemizeZero}

%\item Dr Dean Schneider
%\begin{itemizeZero}
%\item Director of Southern Regional Manufacturing Center
%\item 1617 Research Parkway, College Station, TX 77843, United States
%\item dean.schneider@cesmii.org \;$\bullet$\;  +1.979.458.0251
%\end{itemizeZero}

\end{itemize}


\subsection{Collaborators}
\begin{itemizeZero}
\item Nilay Shah, Imperial College London
\item Anthony Kucernak, Imperial College London
\item Stratos Pistikopoulos and Mahmoud El-Halwagi, Chemical Engineering, Texas A\&M University 
\item Robert Skelton, Aerospace Engineering, Texas A\&M University
\item Satish Bukkapatnam, Industrial Engineering, Texas A\&M University
\item Matthew Cochran, Texas A\&M Institute for Infectious Animal Diseases
\item Donald Chmielewski, Illinois Institute of Technology
\item James Davis and Prakashan Korambath, University of California, Los Angeles
\item Thomas Edgar and Michael Baldea, University of Texas at Austin
\item B. Wayne Bequette, Rensselaer Polytechnic Institute
\item Gerald Parker, Texas A\&M Institute for Infectious Animal Diseases
\item Ricardo Lent, University of Houston
\item Thiago Lopes, University of Sao Paulo
\item Costas Pantelides, Process Systems Enterprise
\item Kiran Sheth, ExxonMobil Research and Engineering
\item Larry Megan and Jesus Flores-Cerrillo, Praxair Inc.
\item Pete Sharpe, Emerson
\item Bill Mock and Ashok Rao, Aspen Technology
\item John Dyck, Haresh Malkani, Howard Goldberg, Miguel Corcio, DOE Smart Manufacturing Institute
\end{itemizeZero}

%====================================================================================

\section*{Biography}
%Amit completed his doctoral studies in chemical engineering from Imperial College London and revising his thesis.
Amit is a subject matter expert on smart manufacturing for the process industry. His research focuses on the development of smart manufacturing platform, products, and technology solutions. He studied chemical engineering at Imperial College, Illinois Institute of Technology and Maharashtra Institute of Technology. He worked as an Industrial Automation Consultant with RasGas Qatar before joining Invensys as a Software Technology Developer for the Advanced Process Control product 'Connoisseur{\texttrademark}’. In academia, he worked as an Assistant Professor of Instrumentation and Control at the College of Engineering Pune. His work underpins enabling technologies for smart manufacturing that focuses on the development of economically and environmentally conscious process design, global optimisation and robust control theory. His seminal contributions on the controller tuning and economic value of control strategies carried out in collaboration with Professor Donald J. Chmielewski has been incorporated in the book titled: 'Smart Process Plants: Software and Hardware Solutions for Accurate Data and Profitable Operations’. His contributions on the grand unification of the industrial automation hierarchy resulted in the development of a state-of-the-art multi-scale fuel cell test platform. He has published numerous scholarly publications and has received several funding grants. He is a recipient of the Excellence in Teaching award at the Illinois Institute of Technology and listed in Marquis Who's who in Science and Engineering. He is a winner of the IChemE Journal's best reviewers award. Currently he is working as an independent research consultant and data analyst.
\label{LastPage}
\end{document}

%fellow with Professor Robert E. Skelton at the Texas A\&M Aerospace Engineering.


%-----------------------------------------------------------------------------------
\par
%=========================================================================
%\clearpage \newpage
%====================================================================================

\vspace{10mm}

%------------------------------------------------------------------------
%\pagestyle{empty} %\pagenumbering{empty} %\setcounter{page}{1}
%------------------------------------------------------------------------
%\def\bibindent{20mm}
%\def\bibname{Bibliography}
%\renewcommand{\bibname}{} %%no name
\renewcommand{\bibsection}{}
\setlength{\bibsep}{18pt}

\bibliographystyle{agsm} %{plainnat}
%Natbib: {ksfh_nat} {agsm} %{unsrtnat} %{plainnat} {acm}

%\bibliography{System_Identification_Control_20190322_0145}
%\addcontentsline{toc}{chapter}{\bibname}
%===============================================================================

%===============================================================================
\end{document}
%===============================================================================

\begin{itemize}
\setlength{\itemsep}{0mm}
\item 2018---2013 \;\; Best Energy Exhibit Award, UK Energy Research Centre
\item 2012---2013 \;\; Best Energy Exhibit Award, UK Energy Research Centre
\item 2011---2012 \;\; IChemE Journals’ Best Reviewers Award
\item 2012---2015 \;\; Engineering and Physical Sciences Research Council (EPSRC) Doctoral Grant
\item 2007---2008 \;\; Invensys Award for Software Development Connoisseur{\texttrademark}% Version  14.3)
\item 2001---2002 \;\; Excellence in Teaching Award, Illinois Institute of Technology
\end{itemize}

%====================================================================================
%====================================================================================
%Manuscripts in preparation or in review
