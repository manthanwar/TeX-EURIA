\textbf{Description of work}\\
\task{2.1}{D1}{}{Development of Non-invasive Diagnostic Techniques for Frescos}{ITP}{DEB}{1}{36}\\
Frescoes, a renowned form of mural art, are delicate and often ancient creations that require meticulous preservation and restoration efforts. Non-invasive diagnostic techniques play a pivotal role in understanding the condition, composition, and underlying issues of frescoes without causing harm to these valuable artworks. Traditional analytical techniques, such as visual examination and manual probing, are often inadequate for unraveling the intricate layers of frescoes. Non-invasive diagnostic methods, including imaging techniques like infrared reflectography (IR) and X-ray radiography, are valuable tools for peering beneath the surface. These techniques can reveal hidden details, underdrawings, alterations, and layers of frescoes without physical intrusion.

\task{2.2}{D1, D2}{M1}{Data Sampling and Analysis of Roman Frescos}{ITV}{DEB, ITP}{1}{36}\\
Frescoes are typically composed of multiple layers, each serving a unique purpose. Frescoes can undergo significant changes due to environmental factors such as humidity, temperature fluctuations, pollution, and exposure to natural light. These environmental stressors can lead to surface deterioration, including cracks, delamination, and pigment fading. Such alterations further obscure the original composition, making it difficult to decipher the artist’s intentions. Restoration efforts, although essential for preserving frescoes, can also introduce complexity. Well-intentioned conservators may apply new layers of plaster, paint, or adhesives to stabilize or repair damaged frescoes. These interventions, while necessary, can complicate the task of identifying and understanding the fresco’s original layers and compositions. This task is similar to the Task 2.3. Here, we will collect the samples of Roman frescos from sites in Italy.

\task{2.3}{D1}{}{Data Sampling and Analysis of Byzantine Frescos}{HUM}{DEB, ITP}{1}{12}\\
Byzantine mosaics are more vibrant, more abstract, adorn walls instead of floors, and feature Christian subjects. We will collect the samples of frescos from the HUM collection and other in the region to be sent to Team ITP for material characterisation in Task 2.1. We will use the application developed in Task 2.2 to then upload the analysed material property data into the material database. We will also carry out standardisation of material property database

\task{2.4}{}{}{Data Sampling and Analysis of Roman Frescos}{ITV}{DEB, ITP}{1}{36}\\
Frescoes are typically composed of multiple layers, each serving a unique purpose. Frescoes can undergo significant changes due to environmental factors such as humidity, temperature fluctuations, pollution, and exposure to natural light. These environmental stressors can lead to surface deterioration, including cracks, delamination, and pigment fading. Such alterations further obscure the original composition, making it difficult to decipher the artist’s intentions. Restoration efforts, although essential for preserving frescoes, can also introduce complexity. Well-intentioned conservators may apply new layers of plaster, paint, or adhesives to stabilize or repair damaged frescoes. These interventions, while necessary, can complicate the task of identifying and understanding the fresco’s original layers and compositions. This task is similar to the Task 2.3. Here, we will collect the samples of Roman frescos from sites in Italy.

\task{2.5}{}{}{Development of Algorithms for Scripts of Late Antiquity on Parchment}{CZP}{DEB}{1}{32}\\
As with many humanistic fields, the skills paleographers claim to have are drawn into question by statisticians, image processing programmers, etc. This the leverage point of our collaborative efforts. It is important to stress that well-known and reliably attributed manuscripts are needed for training the classifier(s). In the example described here, it would be ancient manuscripts such as P66 (a papyrus 2nd Century CE, now in the Bodmer Library, Geneva, Switzerland) or Codex Vaticanus (a vellum manuscript, 340 CE, now in the Vatican).


