\textbf{Description of work}\\
\task{4.1}{D1}{M1}{Cloud Integration of Intangible Cultural Heritage}{DEB}{BAS CZU, SKB}{1}{36}\\
This project will generate a vast amount of data coming from 3 distinct experimental test facilities taking part in this project investigating 3 different aspects at multiple length scales namely (a) material synthesis experiments carried out by MAT, (b) material characterisation experiments carried out by BTU, and (c) process plant testing performed by IEP. In this task we will develop a dynamic map of various data structures. Next, we will contextualize information schema for multidirectional integration of Operations Technology (OT) and Information Technology (IT) exchange also known as Smart Manufacturing in USA or Industry 4.0 in Europe or Advanced Industrial Strategy in UK that will be secured and easily accessible via online systems through a variety of application programming interfaces. Finally, in this task we will develop various data processing, visualisation and analytic tools.

\task{4.2}{}{}{Marian Art in Contemporary Cultural Hyperdulia}{SKB}{DEB}{1}{36}\\
In the task, we will explore how the Virgin Mary has successfully become a pivotal figure of not only Catholic and Eastern Orthodox Christianity but also a broadly addressed agent in the Post-communist Europe. The project comes with the innovative hypothesis that the storytelling is central in successful transmission of Marian devotion, but it needs to be tested. By focusing on Marian storytelling, the project aims at clarifying more general processes, i.e. the nature of contemporary processes of (A1) making-the-history; and (A2) making-thereligion. Based on empirical research and exploratory design of the project a (A3) comprehensive theory and methodology will be developed transferable to other research topics and disciplines.

\task{4.3}{}{}{Virtual Storytelling of Extinct Bosnian Crafts}{BAS}{DEB, CZU, SKB}{1}{36}\\
This task focuses on the digitalisation of the old or extinct Bosnian crafts tradition by exploiting recent advances in AR/VR technologies and further contributing to their preservation. Recently, we developed a virtual museum of the Baščaršija crafts tradition that is facing extinction: četkar (brush maker), kazaz (tailor decorater) and bozadžija (maker of drink called boza) through Virtual Reality (VR) application. This project builds on these ideas to introduce other old or extinct crafts and integrates them in the Eucrite cloud platform ecosystem to extract the benefits of advanced visualization algorithms, integration of latest VR headsets, and develops better more engaging virtual storytelling with integrated innovations in museum aspects. The high immersivity of VR headsets can easily transfer the users into a different place and time, but they need to be offered content that can use the potential of this technology and recreate life in virtual environments. VR video overcomes classical video as a medium and breaks the rules of film language and grammar. As a result users can learn about the historical development of these crafts, get to know meanings of craft names that are not familiar anymore to the general public, and experience the crafts traditions in interactive and engaging VR.

\task{4.4}{}{}{Digitalisation of Bohemian Agricultural Heritage}{CZU}{DEB, BAS SKB}{1}{36}\\
We aim is to create an Augmented and Virtual Reality story of the unique ancestral wine production practices of the small family owned winery located in Moravia region which otherwise would be forgotten. The story will also cover the oral narratives and local traditions of wine utilization for the purposes of wine-based medicine and local cuisine, the community building aspects of wine production and the role of wine in rural festivals and traditions.

\task{4.5}{}{}{Digitalisation of Romani Cultural Heritage}{SKB}{DEB, BAS}{1}{36}\\
This task will duplicate aspects of Task 4.3 and 4.4 and reproduce the results in the context of Romani heritage of Slovakia.


