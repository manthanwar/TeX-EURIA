\textbf{Description of work}\\
\task{1.1}{D1, D2}{M1, M2}{Development of Cloud Platform Ecosystem Tools and Solutions}{DEB}{BEL, CZU, IED, RSB, SKB}{1}{20}\\
In this task we will build the containerised applications and images of necessary back-end infrastructure comprising of high performance compute, network, and storage server nodes that are specifically customised for the cultural heritage actors. These nodes can then be easily initiated and configured by the end-user. In addition, we will develop integrated data visualisation and composable analytics to manage the cultural heritage data and its smart retrieval through vendor agnostic application programming interfaces. Since, some of these types of dataspaces in a variety of forms are already available, e.g. Europeana, DARIAH, ARIADNE, and ARIADNEPlus, etc., we will effectively use them. However, they manage basic data types and lack the advanced cloud-enabled algorithmic tools that we aim to deliver through this work package. These data services are more like basic data of a person including photographs. What we need is a biometric information too to uniquely identify a person. Along these lines we plan to create a unique material property database of frescos in WP2. In this task we will also develop necessary infrastructure to facilitate creation and management of a DNA database of cultural heritage objects.

\task{1.2}{D1}{}{Development of Context-Aware Cultural Informatics}{SKB}{DEB, BEL, CZU, IED, RSB}{1}{36}\\
As discussed earlier, context-driven activities are of paramount importance to cultural heritage research. Take an example of a study carried out in WP4 involving pictures of Virgin Mary shown in Figure ?? in a number of contextual references which can be broadly classified as religiosity-driven spiritual healing, non-religiosity-driven aesthetics, and pragmatic economic or political context. Another example is a study of objects from Biblical antiquities carried out in WP3 where original context, geo-spatiotemporal space-time, historical interpretations,biography of objects, its material properties, human-animal relations, authenticity, integrity, depth, political, and ethical dimensions, etc., In this task, we will focus on contextualising images and develop context-aware cultural informatics that can be easily modified to either analysing Marian hyperdulia in WP4 or pictures from Iraq in WP3. Our methodology will be based on development of document databases with contextual metadata search and retrieval capabilities that are augmented with either intelligent image processing algorithms developed in WP2 or intelligent analytics for recommender system in Task 1.4 below.

\task{1.3}{D1}{}{Development of Enterprise Blockchain Security System}{RSB}{DEB, BEL, CZU, IED, SKB}{1}{36}\\
In this task, we will develop a enterprise level blockchain system for securing cultural heritage assets based on open-source Hyperledger platform. Development will be done at RSB’s Blockchain Technology Laboratory. The blockchain-based system will be private, and therefore does not require high processing power, as computationally-heavy consensus mechanisms can be avoided with Hyperledger. This system will firstly be tested and deployed at partnering cultural heritage institutions, e.g. IED, HUM. Successfully valorised solutions will be made available through open Eucrite cloud marketplace software and application programming interface for securely interconnecting cultural institutions, and using the immutability property of blockchain to correctly track all activities for items connected to the blockchain. Thus, build a mechanism that will ensure trust between different cultural heritage actors and stakeholders.

\task{1.4}{D1, D2}{}{Development of Intelligent Analytics for Technology Enhanced Learning}{BEL}{DEB, CZU, IED, RSB, SKB}{1}{36}\\
Technology enhanced learning analytics in the form of learning dashboards as a specific class of personalised informatics of recommender systems can increase motivation, awareness, reflection, sense-making autonomy, effectiveness, and efficiency of learners and teachers alike, [18]. Dashboards typically capture and visualise traces of learning activities and to enable learners to define goals and track progress towards these goals. Such dashboards provides graphical representations of the current and historical state of a learner or a course to enable flexible decision making either used in traditional face-to-face teaching, online learning, or blended learning settings. However, such recommender systems still have not found major usage in K12 education, [1], and in cultural heritage sector because of unique challenges when used by diverse population and multidisciplinary nature of the cultural heritage. These challenges are further compounded by the increasing role of digital technologies such as developed through various work packages 1 to 4 of this project.


