\textbf{Objectives}\\
This work package will focus on developing two techniques for the analysis of art: (1) experimental analysis of fresco by non-invasive methods; and (2) intelligent algorithms for script and image processing. Frescoes, which are paintings created on wet plaster surfaces, have been a significant form of artistic expression for centuries, dating back to ancient civilizations. They adorn the walls of churches, temples, tombs, and historical sites, offering a glimpse into the artistic and cultural heritage of the past. The preservation and restoration of frescoes are essential to maintaining these valuable artifacts for future generations. Non-invasive diagnostic techniques have become indispensable tools for conservators and art historians in understanding and preserving frescoes. Script is essentially the graphic form of a writing system capable of transcribing any and all utterances of a particular known or unknown language, broadly classified into structure-based and visual-appearance-based techniques. (We do not, for this project, deal with cuneiform scripts, where style is difficult to define.) For theoretical analysis we will rely on the statistical approaches that are mathematically sound, and deliver computing significance of neural network based outcomes, evaluated via the Bayesian statistical approach, via clustering algorithms, and via KDE (kernel density estimation) distributions, [3, 2].
\par

