\item \textbf{DEB:} Materich is a private commercial company (SME) based in Brandenburg, Germany, led by Sabina Ziemian as founder CEO and Amit Manthanwar as founder CTO. Sabina has expertise in material chemistry with over 20 years of academic and industrial research experience gained at Imperial College London, University of Nottingham and Bayer. Amit has expertise in process systems engineering with over 20 years of academic and industrial research, development, teaching, and technology transfer experience gained at  Imperial College London, Texas A\&M University, Illinois Institute of Technology, College of Engineering Pune, the US Smart Manufacturing Institute, Invensys (now Schneider Electric), and RasGas Qatar. He has commercially developed advanced process automation algorithms, tools, and software solutions for economically and environmentally conscious industrial process operations. He has received a number of multidisciplinary consortium-led research grants from the US Department of Energy, the UK Engineering and Physical Sciences Research Council, and the Government of India. Notable awards relevant to this project include the US DOE programme on smart manufacturing. He is the lead investigator for this project. Materich brings a vast amount of experience and expertise to coordinate the project and carry out a number of tasks proposed in this project.

\item \textbf{BAS:} University of Sarajevo is a public research university of Bosnia and Herzegovina. Selma Rizvić is a professor of computer graphics at the university. Her research group focuses on applications of digital technologies to cultural heritage preservation and presentation through Virtual and Augmented Reality, combined with interactive digital storytelling. Selma and her research team will contribute to developing high quality virtual reconstructions of the intangible cultural heritage proposed in WP3.

\item \textbf{BEL:} Katholieke Universiteit Leuven is a research university in Belgium. Katrien Verbert is a professor of computer science at the university. Her research focuses on interactive recommender systems,  visual analytics, explainable AI and applications in learning analytics, healthcare, precision agriculture, media consumption and digital humanities. Robin De Croon is a researcher of intelligent user interfaces. He has developed a suite of information visualization tools for understanding, exploring, explaining and disclosing health information. Katrien and her research group will develop a number of tasks proposed in WP1 and WP2.

\item \textbf{CZP:} Charles University was founded in 1348, making it one of the oldest universities in the world. Yet it is also renowned as a modern, dynamic, cosmopolitan and prestigious institution of higher education. Daniel Říha is an academic researcher active in the field of interactive media since 1998. He is an award-winning interactive media designer (Kunst am Bau, Germany), author and editor in the field of human-computer interaction, computer game studies, interactive media and digital art. He is co-organizer of scientific conference lines for Cyber Hub, Inter-Disciplinary.Net, Oxford, UK (2009-2016) and HCI in Games, HCI International (2019-2024). Hermann Prossinger is a biostatistician who innovated multiple scientific fields by involving analytical methods taken from different field i.e. physics. He is a co-developer of the geometric morphometry and in recent collaboration with Jakub Binter, Silvia Boschetti and Tomas Hladky developed novel methods of analyzing questionnaire data using artificial intelligence. He also uses the visual analysis to evaluate the stimuli to find connections with human perceptual systems.

\item \textbf{CZU:} Czech University is a public research university of Czech Republic. Lubos Smutka, prof., Ing., Ph.D.: The vice for science and research; Head of Department of Trade and Finance; Head of International Center for Rural development Studies, Faculty of Economics and management, Czech university of Life Sciences Prague. An author of more than 372 papers (world trade, European foreign trade, Czech foreign trade, world agricultural production and consumption, agricultural trade etc.), which were presentenced and published in the Czech Republic (local conferences, journals and books) and abroad (conferences, books and journals in Hungary, Ukraine, India, Indonesia, Mexico, Slovakia, France). About 192 respectively 209 papers are recorded in SCOPUS (H-index 17, 1021 citation) respectively WoS database (H-index 16, 682 citations). Participation at eight international conferences and two international seminars as an invited session keynote speaker or invited plenary session keynote speaker. Principal investigator or co-investigator of more than 20 national or international projects.

\item \textbf{HUM:} The Hungarian National Museum founded in 1802 is the national museum for the history, art, and archaeology of Hungary. The archaeological collections of the Museum gather finds from across the whole of historical Hungary and from most of the main archaeological excavations conducted in Hungarian territory. Several Hungarian museums are affiliated to HUM. A number of researchers of the museum as direct beneficiary and in-kind contribution led by their international project coordinator Klaudia Klára Tvergyák will contribute in delivering a number of tasks proposed in this project.

\item \textbf{IED:} Trinity College Dublin is a public research university in Dublin, Ireland. Zuleika Rodgers is Associate Professor in Jewish Studies in the Dept of Near and Middle Eastern Studies and Curator of the Weingreen Museum, Trinity College Dublin. She is a cultural historian who has worked with text and object and whose research spans both antiquity as well as the modern period. Her publications include interdisciplinary edited volumes focussing on text, object, archaeological studies and reception history.  She has been curator of the Weingreen Museum since 1999 and has run workshops with the collection for school students, university students and the public. She has organised a number of exhibitions and also directed a project on digitising the museum’s catalogue. Relevant to this project is her museum work, her teaching of the Near East and Mediterranean World from antiquity to the present and her work as a cultural historian who examines antiquity and its reception. Christine Morris is the Andrew A David Professor of Greek Archaeology and History in the Department of Classics, Trinity College Dublin. She is a specialist in Mediterranean archaeology, working in particular on Crete and Cyprus. Her current projects include the Atsipadhes Archaeological Project; East Cretan Peak Sanctuaries Project (ECPSP) (both with Alan Peatfield), and she is co-PI on the interdisciplinary project, 'The Many Lives of a Snake Goddess'. Her co-edited books include Ancient Goddesses (1998), An Archaeology of Spiritualities (2009), Unlocking Sacred Landscapes: Spatial Analysis of Ritual and Cult (2019).  Relevant to this project is her expertise in ceramic studies (pottery and figurines); object biography; museum cataloguing; and her collaborative work on 3D scanning and printing in relation to Cretan Bronze Age figurines. Zuleika and Christine will focused on the digitalisation of the Weingreen Museum through this project by delivering a number of proposed tasks.

\item \textbf{ITP:} University of Pisa is a public research university in Pisa, Italy. Marco Lezzerini is a Professor of mineralogy. He and his research group focuses on the mineralogical-petrographic analysis for the environment and cultural heritage. The group deliver WP2.

\item \textbf{ITR:} The American University of Rome is a private Liberal Arts college offering Bachelors and Masters degrees accredited by Middle States Commission on Higher Education. The program in Cultural Heritage, forms part of the Graduate School, and offers high quality, postgraduate education comprising the latest skills to students aiming to work as cultural heritage professionals. The program has a particular emphasis on community work, sustainability, development, conflict and post conflict heritage protection and peacebuilding and antiquities crime prevention. Valerie Higgins is Program Director for Cultural Heritage and Associate Professor of Archaeology. Her research covers the areas of antiquities crime prevention, community engagement, heritage tourism and conflict and post-conflict heritage protection. She is widely published in these areas and in addition has taught courses and chaired meetings with a wide variety of international partners including International Organizations, NGOs, EU, and military organizations charged with Cultural Property Protection. She and her research team will be developing WP5.

\item \textbf{ITV:} University of Tuscia is a young university in Viterbo, Italy. Alessandra Bravi is a classical archaeologist. She studied archaeology in Rome and Heidelberg. She has carried out research at the Universities of Heidelberg, where she qualified in Classical Archaeology, and Perugia, and for the past three years has been working as a scientist and researcher at Tuscia University. Her research focusses on six larger areas: 1) Interaction between written and visual culture; 2) cultural transfer of visual and material culture; 3) resistance against Greco-Roman imperialism and the cultural identity of ‘marginal’ societies: Diaspora, Asia Minor, Roman Egypt etc. 3) pragmatic interpretation of art based on the interaction between artwork and the society of user-viewers; 4) Changes in late antique society and its visual culture and the process of de-sacralizing of classical art; 5) the social role of art and material culture as indicators of status, cultural distinction, and prestige 6) The role and function of the Classical Cultural Heritage in Byzantine culture. She will be delivering digitalisation of the museum of Amelia.

\item \textbf{RSB:} Belgrade Metropolitan University is a research university in Belgrade, Serbia. Nemanja Zdravković completed his M.Sc. in electrical engineering and computer science, scientific field telecommunications, at the Faculty of Electronic Engineering, University of Niš in 2012, and his Ph.D. studies at the Norwegian University of Science and Technology (NTNU) in Trondheim, Norway, in 2017, as well as at the University of Niš in 2017, for which he has received a dual Ph.D. degree. Dr. Zdravković has been an Assistant Professor at the Faculty of Information Technologies at Belgrade Metropolitan University and since 2020 head of the Blockchain Technology Laboratory at BMU, and since April 2023 he is the Dean of the Faculty of Information Technology at BMU. Besides teaching activities in the field of computer networks, blockchain technology and computer architecture, he conducts research which includes the application of blockchain technology in healthcare, cooperative and distributed network analysis, RF and optical telecommunication systems analysis and information theory. Dr. Zdravković is a member of the Institute of Electrical and Electronics Engineers (IEEE) and serves as a reviewer of the flagship conferences in wireless telecommunication systems.

\item \textbf{SKB:} The Institute of Ethnology and Social Anthropology at the Slovak Academy of Sciences in Bratislava is a prominent institution for ethnological, socio-anthropological, and religious research in Slovakia and Central Europe. This research team will be led by Tatiana Podolinska who is the director of the Institute director. Her research work focuses on contemporary religiosity and spirituality, with a focus on Romani studies and Marian devotion in Europe. The Academy has a vast collection of cultural heritage some of which has already been digitalised. This project will extend the work carried out by Tatiana and her colleagues Elena Marushiakova, Lubica Volanska, and Andrej Gogora. They will contribute in developing various tasks in this project. Elena Marushiakova works at the Institute of Ethnology and Social Anthropology of the SAS from 2023. For the past eight years, she worked at St Andrews University (UK), initially for one year as a Leverhulme visiting professor and then as a research professor and Principal Investigator of the ERC Advanced Grant 2015, Nr. 694656 "Roma Civic Emancipation Between the Two World Wars". From 2001 to 2004, she worked at the Institute of Ethnology of Leipzig University in Germany as part of the SFB Difference and Integration project. Economical Symbiosis and Cultural Separation: Service Nomads in Rural and Urban Contexts. She was also employed at the Institute of Ethnology and Folklore Studies with the Ethnographic Museum of the Bulgarian Academy of Sciences. Andrej Gogora is a researcher at Institute and Head of the Scientific Collections Department. His research focuses on the methodological foundations of digital humanities. Currently, he is involved in building digital text and image resources in humanities, in particular, digital collections of ethnological research reports. He is the coordinator on behalf of the IESA SAS as a cooperating partner of the European research infrastructure DARIAH-EU.

