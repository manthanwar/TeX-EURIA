%\input{impact-guidelines.tex}
%==============================================================================
\clearpage
%------------------------------------------------------------------------------
\section{Impact}
\subsection{Innovation potential}
The unique contributions of this work are:
\begin{enumerate}[nosep]
\item The Key Performance Indicators (KPI) to analyse impact of cultural heritage on digitalisation of museums, regional development, tourism, and enhanced learning abilities through a unique cloud-enabled solutions
\item Algorithms and tools for data gathering, processing, visualization, and intelligent  analytics for script, images, 3D objects, and intangible heritage.
\item Open software for museum information management
\item Open software for knowledge management
\item Open software for material properties
\item Open software for enterprise Blockchain security
\item Number of open user friendly software apps based on the case studies developed
\item Highly functional and content rich web portal for secured information access, training, and relationship management.
\end{enumerate}


\subsection{Potential  impact}
The impact of this work to be in several areas:

\begin{itemize}[nosep]
\item The proposed methodology will be open to the public by means of publications and events directly oriented to the public; we will cover both the channels of scientific dissemination and dissemination to the wider public with different media and actions. Finally, the Consortium will endorse as much as possible the open source policy and will make efforts for granting use of the basic technology developed within this project to the external world. This will become a concrete possibility for external potential users both at the level of the basic assets and of the tools developed for the design and implementation of VR installations. We expect the impact to the works for multidisciplinary researchers from computer science, visual arts, literature, film directing, psychology, communicology and human computer interaction. We believe that this work could be used also in other areas of life, such as education and journalism. The workflows developed can be reused in various other applications of cultural heritage carried  out in this project and beyond.


\end{itemize}

	1. This study will improve the short and long term efficiency of the ECCCH platform and provide a tool for immediate feedback .
	2. The survey study will also contribute to making the stakeholders be aware of, have access to and use the ECCCH platform, its tools and services for the study, digitization, conservation, valorization and access to cultural heritage artifacts and related data, in particular for the sharing and preservation of such data, and are involved in its validation and assessment, in view of continuously improving the ECCCH’s performance and use. The impact will consider gender aspects.
	3. These efforts would help to establish a pan-European network of key stakeholders from cultural heritage institutions, including a robust scientific and professional community and ensure openness to the cooperative efforts of a wide community of users.
The results of the study will be disseminated to the stakeholders and research community.

\subsubsection{Target groups}\label{subsubsec-Target}
The target groups for the work undertaken are:
\begin{enumerate}[nosep]
\item Education and research  institutions (academic  community)
\item Galleries, Libraries,  Archives, and Museums (GLAM)  institutions
\item Creative and cultural industry, NGOs
\item Churches and other religious institutions
\item Municipalities, government ministries and policymakers,
\item Tourists, especially interested in intangible cultural heritage
\item Citizens, residents, communities Digital and media content users, consumers
\item Young generations, students.
\item government agencies
\end{enumerate}

\subsubsection{Academic impact}
This project will encourage academics adapt proposed skill development techniques and assessment frameworks in their own practice. Academic researchers will be able to further evolve proposed cloud framework and architecture developed in this work to develop similar systems for other areas or develop tools that can be deployed in the Eucrite cloud ecosystem.


\subsubsection{Societal impact}
The rural community like the Amelia where the proposed work carried out will have a significant impact in its overall aspect of regional economic development including promotion of tourism, skill development, and youth empowerment. This community and Umbria region of Italy will be directly impacted by this project. The international workshop hosted in this rural community will bring this region to the international spotlight and help grow the local economy, thus bring direct socioeconomic benefit to the local population.
%==============================================================================
%------------------------------------------------------------------------------

\subsection{Scale and measure of impact}
Using WP1 Task 1.6 we will review currently available best practices as measures of Key Performance Indicators. This will be carried out via a systematic bibliometric analysis and pan-European empirical study. Next, we will develop benchmarks and use those measures to benchmark our target groups listed in section \ref{subsubsec-Target}. in at least four low-to middle-income countries Bosnia, Czechia, Slovakia, Hungary we aim to demonstrate a significant improvements in these numbers to set a new benchmark as we improve our knowledge delivery mechanisms. We will train at least 100 young students, 50 teachers, 50 researchers, 100 museum staff, 10 cultural ministry administrators,  through the workshops and activities carried out in WP5. Based on merit we will be awarding at least 40 new scholarships to students across Europe, especially yough girls and students from disadvantaged backgrounds. We will also be publishing scholarly publications to report our findings. The number of users as traffic on our state-of-the-art software platform will add intellectual and promotional value to our work.

\subsubsection{Measures to maximise impact}
We will adapt agile project development strategy to maximise impact. By utilizing adaptive learning methods we will focus on improving the quality of our instructions and course material. In addition, some of the following aspects will further help us maximize impact:
\begin{enumerate}[nosep,left=1mm]
\item A wider review of best practices and better understanding to enhanced knowledge base required to monitor and assess impact including forward looking approaches aimed to anticipate and prepare for future or emerging technology challenges.
\item Apply effective risk assessment and risk management strategies to deliver project outcomes.
\item Increasing the EU knowledge base and guidance on measures to manage quality and evidence for policy-making, planning and implementation.
\item Science and evidence-based implementation of the European Green Deal and the Sustainable Development Goals, notably the SDP 5: gender equality,  SDP 10: reduced inequality.
\end{enumerate}
%==============================================================================
%------------------------------------------------------------------------------
\subsection{Communication and Dissemination}
\subsubsection{Target beneficiaries}
We have identified the following beneficiaries for the knowledge transfer:
\begin{enumerate}[nosep,left=1mm]
\item Multidisciplinary academic researchers:
\begin{enumerate}[nosep]
\item The art researchers will advance their understanding on how mathematical algorithms of art can be developed. They will learn to use techniques developed to undertake new CH research. Open cloud will also promote solution development for their specific problems.
\item The science and engineering researchers will be able to utilize and integrate the algorithms and tools in their analysis to create innovative cost effective solutions in their areas. For example, 3D Vision algorithm can be used by Industry 4.0 and robotics community. New material properties database can be used by industry to build new solutions for preservation, restoration, and reconstruction.
\end{enumerate}
2. Institutional stakeholders:
\begin{enumerate}[nosep]
\item The data provided and software innovation carried out in this work will allow small entrepreneurial museums as well as large museums to develop and market new solutions, new exhibitions, designed to solve specific heritage problems.
\end{enumerate}
\item Government agencies:
\begin{enumerate}[nosep]
\item This project will promote European sense of belonging by achieving cultural cohesion and resiliency through a shared digitalisation journey
\item highly skilled workforce developed through evidence-based policy guidelines developed will contribute to the sustainable innovation potential of rural and urban communities.
\item The best practices and key performance indicators developed through this project will enable to predict potential areas of improvements and policy interventions. We believe that our proposed cloud ecosystem will help government administrators and local agencies to visualise and communicate skill gap information more effectively, they will be able to integrate their museum collections and develop a national database in an hierarchical manner that goes all the way down to a rural community. Agencies role in the gender-responsive youth empowerment and rural development will be further strengthened by this work.
\end{enumerate}
\end{enumerate}

\subsubsection{Dissemination}
We will have regular progress reports updates on the project which will immediately be made available to a larger audience via our online Platform. Additionally, we will be publishing scientific papers and presenting our findings through conference presentations delivered on a wide variety of topics to both European and international audiences. Through our onsite workshops as well as various online mediums we will be disseminating our new knowledge to have a maximum impact for the cultural heritage and literacy.

The outcomes of this work will be widely disseminated among all relevant stakeholders  to achieve impact at national and European level. Periodic quality monitoring to track progress towards achieving all the agreed EU-level targets and indicators will also take place through the meetings among the project participants and relevant stakeholders. At the end of the first cycle/year, we revisit the set of priority areas in order to adjust them or set new ones for the following cycle/year, based on current challenges and to reflect on the progress made. We will publish regular progress reports taking stock and evaluating the achievements made. In the mid cycle we will also organise a mid- term review event.

Education and training have a vital role to play in shaping the future of European landscape. It is widely recognised that there are many barriers to effective adaptation of climate mitigation strategies that we must overcome. Some of these aspects include awareness of climate risks and the complexity of adapting to future technology solutions. In order to bring down barriers to learning and improve access to quality education, to help develop workforce skills in a rapidly changing technology landscape, we intend to develop and offer technology and knowledge transfer programme in WP3 in the form ``micro-credentials'' to the targeted individuals. Our training shall involve flexible acquisition and recognition of knowledge, skills and competencies to meet new and emerging needs in society and labour market as a compliment to traditional means of learning. We intend to provide training using many different formats: formal, non-formal and informal learning settings so as to enable learner acquire the knowledge, skills and competencies they need to thrive in an changing labour market, adapting to and using newly discovered breakthroughs carried out in this project.


\subsubsection{Communication}
We plan to make all results conducted in this project openly available by publishing them in scholarly publications. The overall aim is to increase social awareness and understanding of role of art in computer education. We want to educate a new generation of artists, scientists, academics, involved in the art of computer programming and the computer programming of art. We dedicated the whole WP5 for the preparation of appropriate teaching materials and for communication promoting the project findings to relevant stakeholders via workshops. Along with our proposed information platform in WP5, and by using various other social/professional media platforms, we will make our content and newly discovered knowledge available to wider audience

The COVID-19 pandemic has brought unprecedented challenges and opportunities for education and training systems. Utilizing the impact of digital technologies we aim to bridge the connectivity gaps within urban-rural settings, while also highlighting the potential of education and training to build resilience and foster sustainable and inclusive growth. Thus contributing to the society and economy become more cohesive, inclusive, digital, sustainable, green and resilient, and for citizens to find personal fulfilment and well-being, to be prepared to adapt and perform on a changing labour market and to engage in active and responsible citizenship.

The periodic monitoring of progress towards the set objectives through systematic collection and analysis of internationally comparable data will be utilised to provide high end classroom type and hands-on learning by doing skills. Towards that aim, digital technologies play an important part in making learning environments, learning materials and teaching methods adaptable and suitable to diverse learners. They can promote genuine inclusion – provided that digital gap issues, both in terms of infrastructure and of digital skills, are addressed in parallel. This very issue is at the heart of our proposed digital learning management and technology platform ecosystem developed in WP1 and WP5. The impact of WP2 is to make training systems become more flexible, resilient, future-proof and appealing, reaching out to a more diverse groups by providing upskilling and reskilling training opportunities in making the green and digital transitions to an environmentally sustainable, circular and climate-neutral economy.


\subsubsection{Exploitation}
Although our newly discovered knowledge will be made available in an open format to many of our stakeholders, we believe that there is a appetite for its commercial exploitation through our industrial partners. We will commercially exploit developed knowledge to be passed on to industry professionals via paid training workshops. We believe there will be interest across the industry sectors since art, architecture and data visualization plays significant role in multiple industry sectors. In addition, other avenues for commercial exploitation will be finalized in the early months of the project. Finally, through a multi-tier membership model we will charge annual membership fees for museums and other actors to join the foundation and avail the benefits at a reduced rate. Assuming 4 levels of memberships with fees anywhere between EUR 0 and EUR 20k we can provide technology solutions are affordable rates to museums big and small.

%\subsubsection{Intellectual property management strategy}
%All results will be evaluated for the potential intellectual property before publication. Results relevant to a generation of intellectual property for commercialization of material ad process technology shall be protected by patent applications. The precise rules of handling data publication and intellectual property will be discussed in the consortium agreement agreed amongst the team members and the EU.
%==============================================================================
%------------------------------------------------------------------------------
